\documentclass[11pt]{article}
\usepackage{amsmath,amssymb,amsfonts}
\usepackage{geometry}
\geometry{margin=1in}
\usepackage{cite}

\title{Active Spectroscopy for Stratospheric Aerosol Injection: Information-Optimal Probing and Control-Oriented Surrogates}

\author{AI-Generated Research Paper}
\date{\today}

\begin{document}
\maketitle

% === AREA: Stratospheric aerosol injection (SAI) climate impacts and uncertainties. ===

\subsection{Active spectroscopy methodology: information-optimal probing, frequency-domain identification, and control-oriented surrogates}

We formalize pulsed SAI as an active-sensing experiment designed to extract the climate system's frequency-dependent gains, lags, and nonlinear distortions with minimal intrusion. Let the control input be the global-equivalent sulfur injection rate, decomposed as

\begin{equation}
u(t) \;=\; u_0 \;+\; \delta u(t), \qquad \mathrm{rms}(\delta u)\in[0.05,0.5]\ \mathrm{TgS\,yr^{-1}},
\end{equation}
superimposed on a steady baseline $u_0$ (e.g., $5\ \mathrm{TgS\,yr^{-1}}$). Outputs include global-mean surface temperature (GMST), regional precipitation indices, circulation metrics, and stratospheric aerosol optical depth (SAOD), all routinely simulated in intermediate and comprehensive Earth system models with interactive aerosols and chemistry \cite{tilmes2018stratospheric,richter2022assessingresponses,kravitz2011thegeoengineering}. The central premise is that frequency-domain system identification under designed perturbations is the most direct and data-efficient approach to learn control-relevant dynamics \cite{ljung1999system,pintelon2012system} and to support closed-loop SAI design \cite{macmartin2014dynamicsof}.

\paragraph{Information-optimal probe design}
We excite the system with multi-sine, chirp, and pseudo-random binary sequences (PRBS) over monthly-to-interannual bands ($f_k\in[0.02,2]$ cycles yr$^{-1}$):

\begin{equation}
\delta u(t) \;=\; \sum_{k=1}^{K} a_k \sin\!\big(2\pi f_k t + \phi_k\big) \;+\; \mathrm{PRBS}(t) \;+\; \mathrm{chirp}(t).
\end{equation}
Phases $\{\phi_k\}$ are chosen to minimize the crest factor $CF(\delta u)=\|\delta u\|_\infty/\mathrm{rms}(\delta u)$ (reducing instantaneous extremes), while amplitudes $\{a_k\}$ target bands with favorable signal-to-noise ratio (SNR) under internal variability. We cast input design as a constrained Fisher-information maximization \cite{ljung1999system,pintelon2012system}:

\begin{align}
\max_{\{a_k,\phi_k\}} \;\; & \Phi\!\left(\mathcal{I}(\theta)\right) \quad \text{s.t.} \quad \mathrm{rms}(\delta u)\leq A_{\max},\ \ CF(\delta u)\leq C_{\max}, \nonumber\\
& \mathrm{SAOD}(t) \leq \bar{S}, \quad \text{additional protocol constraints},
\label{eq:input_design}
\end{align}
where $\Phi$ is, e.g., $\log\det\mathcal{I}$ admits a closed form summing per-frequency sensitivities weighted by $S_{uu}(\omega_k)$ and noise spectra \cite{pintelon2012system}.

\paragraph{Uncertainty quantification and robust control hooks}
Emulator uncertainty is quantified by (i) asymptotic covariance from spectral regression, (ii) bootstrap across ensemble members and segments, and (iii) residual coherence tests \cite{pintelon2012system}. We map these to frequency-dependent uncertainty envelopes $\Delta G_i(\omega)$ and parametric sets for \eqref{eq:linear_predictor}. Robust controllers (e.g., $\mathcal{H}_\infty$ or constraint-based MPC) then seek to track global-mean temperature while bounding regional hydrological variance and SAOD excursions against worst-case emulators within these sets, extending prior closed-loop analyses of geoengineering \cite{macmartin2014dynamicsof}.

\paragraph{Validation and cross-model transfer}
We validate emulators hierarchically: aquaplanet or EMIC for rapid iteration, then intermediate and WACCM-like configurations with interactive aerosols \cite{tilmes2018stratospheric, richter2022assessingresponses}. Cross-model transfer is assessed by training the emulator on one tier and evaluating prediction and control performance on another (including GeoMIP-style protocols \cite{kravitz2011thegeoengineering}). Metrics include NRMSE, magnitude/phase errors, coherence, HDI, and closed-loop regret relative to an oracle with access to the full ESM.

\section{Multi-objective optimization for robust SAI control and information-optimal probing}

Stratospheric aerosol injection (SAI) entails simultaneous objectives and constraints across globally coupled subsystems: tracking a mean temperature target, limiting regional hydrological disruption, bounding aerosol optical depth and chemistry-related risks, and moderating actuation variability to avoid excitation of internal modes (e.g., QBO, monsoons) \cite{robock200820reasons, kravitz2011thegeoengineering, tilmes2018stratospheric, richter2022assessingresponses}. These trade-offs argue for an explicitly multi-objective control formulation layered on top of identified surrogate dynamics (Sec. E1–E2) and for probe designs that maximize information per unit perturbation subject to environmental limits. We leverage standard results from frequency-domain system identification \cite{ljung1999system, pintelon2012system} and optimal input design based on Fisher information \cite{afroz2023model} to parameterize uncertainty, and synthesize controllers using a combination of multi-objective MPC and H-infinity shaping that are robust across model discrepancies and internal variability \cite{macmartin2014dynamicsof, tomasgard2012optimal}.

We define the manipulated variable as the zonal-mean SAI injection rate $u(k)$ (TgS yr$^{-1}$), and outputs $y(k) = [T_{\mathrm{GMST}}(k), P_{\mathrm{Sahel}}(k), \mathrm{SAOD}(k), \ldots]^\top$ at monthly resolution. From E1–E2, we obtain a linear surrogate predictor over a receding horizon, either as an ARX/ARMAX model with frequency-calibrated coefficients or as a Koopman-regressed linear time-invariant predictor,

\begin{equation}
x(k+1) = A_i x(k) + B_i u(k) + E_i w(k), \quad y(k) = C_i x(k) + D_i u(k),
\end{equation}
indexed by $i \in \mathcal{M}$ to represent an uncertainty set spanning bootstrap realizations, identification posteriors, and cross-model surrogates for robustness (E5). The index set $\mathcal{M}$ encodes polytopic or multi-model uncertainty propagated from spectral-regression covariance and Fisher information (Sec. E1) \cite{pintelon2012system, cox2020climate}.

We pose a vector of performance metrics over an MPC horizon $N$:
\[
J_{\mathrm{vec}} \equiv \Big(\, \|e_T\|_2, \ \mathrm{Var}[P], \ \|a\|_2, \ \mathrm{TV}(u), \ \mathrm{HDI}, \ \Phi_{\mathrm{lag}} \,\Big),
\]
where $e_T(k) = T_{\mathrm{GMST}}(k) - r_T(k)$, $\mathrm{Var}[P]$ is the predicted variance of regional precipitation indices under disturbance ensembles, $a(k)$ proxies SAOD excursions above a policy band, $\mathrm{TV}(u) = \sum_k |u(k)-u(k-1)|$ penalizes abrupt pulsing, $\mathrm{HDI}$ is a harmonic distortion index derived from second-order Volterra terms (E2), and $\Phi_{\mathrm{lag}}$ measures band-limited phase lag relative to the identified frequency response.

Here, CVaR at level $\beta$ penalizes the tail of regional precipitation anomalies to explicitly manage extremes (consistent with concerns about extremes under SAI \cite{chen2023data}). Safety constraints include envelopes on SAOD and phase-lag margins in sentinel bands used for early warning (E4). The robust constraints are enforced scenario-wise across $i \in \mathcal{M}$ (multi-model robustness; E5) and across disturbance samples $w(k)$ capturing ENSO-like variability and volcanic perturbations (E3). Problem (P) reduces to a convex quadratic program with linear inequalities for linear surrogates and convex risk measures; nonconvexities from HDI and phase-lag can be handled by successive convexification around the current operating point and by linearizing frequency-domain objectives.

\section{Control synthesis and experimental protocols}

where $S_i(z) = (I + G_i(z)K(z))^{-1}$ and $T_i(z) = G_i(z)K(z)S_i(z)$ are the sensitivity and complementary sensitivity with respect to plant $G_i$ identified from E1-E2; $W_S$, $W_T$ and $W_U$ are frequency weights derived from the measured gains, phase margins, and coherence to encode performance-robustness targets (e.g., attenuate interannual bands to protect monsoons while allowing low-frequency tracking). Multi-model H-infinity synthesis proceeds via standard LMI relaxations over the vertices of $\mathcal{M}$ or via DK iteration, and can be used either as a standalone controller or as a stabilizing inner loop beneath the multi-objective MPC.

\section{Conclusion}

This research contributes to the broader goal of developing scientifically rigorous, ethically responsible approaches to climate intervention that can inform policy decisions while maintaining the highest standards of safety and environmental protection.

\nocite{afroz2023model,annan2010optimal,bathiany2020early,broccoli2021impact,chattopadhyay2020data,chattopadhyay2020reduced,chen2023data,cox2020climate,feng2021optimal,keith2010researchon,keith2016optimal,kravitz2011thegeoengineering,lazaro2020robust,ljung1999system,lucarini2022data,macmartin2014dynamicsof,macmartin2014model,macmartin2014optimal,pintelon2012system,richter2022assessingresponses,ricke2021impact,robock200820reasons,style2016control,tilmes2018stratospheric,tomasgard2012optimal,visioni2021climate,visioni2021robust}

\bibliographystyle{plain}
\bibliography{../input/references_complete}

\end{document}
